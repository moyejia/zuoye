%% 该模板修改自《计算机学报》latex 模板
%% 主要是将双栏改成单栏,去掉了部分计算机学报标识;
%% 源文件自:https://www.overleaf.com/latex/templates/latextemplet-cjc-xelatex/ybmmymncrrmw
%% 
%%
%% This is file `CjC_template_tex.tex',
%% is modified by Zhi Wang (zhiwang@ieee.org) based on the template 
%% provided by Chinese Journal of Computers (http://cjc.ict.ac.cn/).
%%
%% This version is capable with Overleaf (XeLaTeX).
%%
%% Update date: 2023/03/10
%% -------------------------------------------------------------------
%% Copyright (C) 2016--2023 
%% -------------------------------------------------------------------
%% This file may be distributed and/or modified under the
%% conditions of the LaTeX Project Public License, either version 1.3c
%% of this license or (at your option) any later version.
%% The latest version of this license is in
%%    https://www.latex-project.org/lppl.txt
%% and version 1.3c or later is part of all distributions of LaTeX
%% version 2008 or later.
%% -------------------------------------------------------------------

\documentclass[10.5pt,compsoc,UTF8]{CjC}
\usepackage{CTEX}
\usepackage{graphicx}
\usepackage{footmisc}
\usepackage{subfigure}
\usepackage{url}
\usepackage{multirow}
\usepackage{multicol}
\usepackage[noadjust]{cite}
\usepackage{amsmath,amsthm}
\usepackage{amssymb,amsfonts}
\usepackage{booktabs}
\usepackage{color}
\usepackage{ccaption}
\usepackage{booktabs}
\usepackage{float}
\usepackage{fancyhdr}
\usepackage{caption}
\usepackage{xcolor,stfloats}
\usepackage{comment}
\setcounter{page}{1}
\graphicspath{{figures/}}
\usepackage{cuted}%flushend,
\usepackage{captionhack}
\usepackage{epstopdf}
\usepackage{gbt7714}

%===============================%

\headevenname{\mbox{\quad} \hfill  \mbox{\zihao{-5}{ \hfill 机器学习  } \hspace {50mm} \mbox{2024 年 5 月}}}%
\headoddname{韩海宇 \hfill 练习}%

%footnote use of *
\renewcommand{\thefootnote}{\fnsymbol{footnote}}
\setcounter{footnote}{0}
\renewcommand\footnotelayout{\zihao{5-}}

\newtheoremstyle{mystyle}{0pt}{0pt}{\normalfont}{1em}{\bf}{}{1em}{}
\theoremstyle{mystyle}
\renewcommand\figurename{figure~}
\renewcommand{\thesubfigure}{(\alph{subfigure})}
\newcommand{\upcite}[1]{\textsuperscript{\cite{#1}}}
\renewcommand{\labelenumi}{(\arabic{enumi})}
\newcommand{\tabincell}[2]{\begin{tabular}{@{}#1@{}}#2\end{tabular}}
\newcommand{\abc}{\color{white}\vrule width 2pt}
\renewcommand{\bibsection}{}
\makeatletter
\renewcommand{\@biblabel}[1]{[#1]\hfill}
\makeatother
\setlength\parindent{2em}
%\renewcommand{\hth}{\begin{CJK*}{UTF8}{gbsn}}
%\renewcommand{\htss}{\begin{CJK*}{UTF8}{gbsn}}


\begin{document}

\hyphenpenalty=50000
\makeatletter
\newcommand\mysmall{\@setfontsize\mysmall{7}{9.5}}
\newenvironment{tablehere}
  {\def\@captype{table}}

\let\temp\footnote
\renewcommand \footnote[1]{\temp{\zihao{-5}#1}}


\thispagestyle{plain}%
\thispagestyle{empty}%
\pagestyle{CjCheadings}

% \begin{table*}[!t]
\vspace {-13mm}


\onecolumn
\zihao{5-}\noindent 韩海宇 \hfill 机器学习\hfill 2024 年 5 月\\
\noindent\rule[0.25\baselineskip]{\textwidth}{1pt}

% \onecolumn
% \zihao{5-}\noindent 第??卷\quad 第?期 \hfill 计\quad 算\quad 机\quad 学\quad 报\hfill Vol. ??  No. ?\\
% \zihao{5-}
% 20??年?月 \hfill CHINESE JOURNAL OF COMPUTERS \hfill ???. 20??\\ 
% \noindent\rule[0.25\baselineskip]{\textwidth}{1pt}

{
\centering
\vspace {11mm}
{\zihao{2} \heiti 练习}

\vskip 5mm

{\zihao{4}\fangsong 韩海宇}

% \footnote{\noindent \zihao{6} 
% % 收稿日期:\quad \quad -\quad -\quad ;最终修改稿收到日期:\quad \quad -\quad -\quad .*投稿时不填写此项*. 本课题得到… …基金中文完整名称(No.项目号)、… …基金中文完整名称(No.项目号)、… … 基金中文完整名称(No.项目号)资助.
% \textsf{作者名1},学号,学位(或目前学历),主要研究领域为*****、****. E-mail: **************.\textsf{作者名2},学号,学位(或目前学历),主要研究领域为*****、****. E-mail: **************. \textsf{作者名3},学号,学位(或目前学历),主要研究领域为*****、****. E-mail: **************.}





% \zihao{6}{\textsf{论文定稿后,作者署名、单位无特殊情况不能变更。若变更,须提交签章申请,国家为中国可以不写,省会城市不写省名,其他国家必须写国家名。}}
}

\vskip 5mm


\section{\heiti 题目要求}

在总共200人的样本中: 有10人事实上患有A病(阳性);经过检测后,9人判定患有A病,而1人判定并不患有A病;另外的190人实际上并不患有A病(阴性),经过检测后,其中的10人被判定患有A病,另外的180人判定不患有A病。

计算精度、错误率、特异度、敏感度。
\section{\heiti 解答}

TP(True Positive)表示真阳性,FP(False Positive)表示假阳性,TN(True Negative)表示真阴性,FN(False Negative)表示假阴性。

根据给定的数据:

10人实际上患有A病(阳性),其中9人被正确判定患病(TP = 9),1人被错误判定为不患病(FN = 1)。

190人实际上不患有A病(阴性),其中180人被正确判定不患病(TN = 180),10人被错误判定为患病(FP = 10)。

我们可以代入公式计算各个指标:

\textbf{精度(Accuracy):}表示正确预测的比例。

\[
\text{Accuracy} = \frac{\text{TP} + \text{TN}}{\text{TP} + \text{TN} + \text{FP} + \text{FN}} = \frac{9 + 180}{9 + 180 + 10 + 1} = \frac{189}{200} = 0.945
\]

\textbf{错误率(Error Rate):}表示错误预测的比例。

\[
\text{Error Rate} = \frac{\text{FP} + \text{FN}}{\text{TP} + \text{TN} + \text{FP} + \text{FN}} = \frac{10 + 1}{9 + 180 + 10 + 1} = \frac{11}{200} = 0.055
\]


\textbf{特异度(Specificity):}表示正确识别阴性样本的比例。

\[
\text{Specificity} = \frac{\text{TN}}{\text{TN} + \text{FP}} = \frac{180}{180 + 10} = \frac{180}{190} \approx 0.947
\]


\textbf{敏感度(Sensitivity):}表示正确识别阳性样本的比例。

\[
\text{Sensitivity} = \frac{\text{TP}}{\text{TP} + \text{FN}} = \frac{9}{9 + 1} = \frac{9}{10} = 0.9
\]

\section{\heiti 总结}

在这项研究中,包含200人的样本中,有10人实际患有A病。检测结果显示,这10人中有9人被正确判定为阳性,1人被误判为阴性。同时,在实际不患病的190人中,有180人被正确判定为阴性,10人被误判为阳性。基于这些数据,分类器的精度为94.5$\%$,即在所有样本中有94.5$\%$的人被正确分类。错误率为5.5$\%$,表示有5.5$\%$的样本被错误分类。特异度为94.7$\%$,意味着在所有实际不患病的人中,有94.7$\%$被正确识别为阴性。敏感度为90$\%$,表示在所有实际患病的人中,有90$\%$被正确识别为阳性。这些结果表明,分类器在识别实际阴性样本方面表现良好,同时在识别实际阳性样本方面也有较高的准确性。

\end{document}


